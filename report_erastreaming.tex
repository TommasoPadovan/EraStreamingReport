
%% bare_conf.tex
%% V1.4b
%% 2015/08/26
%% by Michael Shell
%% See:
%% http://www.michaelshell.org/
%% for current contact information.
%%
%% This is a skeleton file demonstrating the use of IEEEtran.cls
%% (requires IEEEtran.cls version 1.8b or later) with an IEEE
%% conference paper.
%%
%% Support sites:
%% http://www.michaelshell.org/tex/ieeetran/
%% http://www.ctan.org/pkg/ieeetran
%% and
%% http://www.ieee.org/

%%*************************************************************************
%% Legal Notice:
%% This code is offered as-is without any warranty either expressed or
%% implied; without even the implied warranty of MERCHANTABILITY or
%% FITNESS FOR A PARTICULAR PURPOSE! 
%% User assumes all risk.
%% In no event shall the IEEE or any contributor to this code be liable for
%% any damages or losses, including, but not limited to, incidental,
%% consequential, or any other damages, resulting from the use or misuse
%% of any information contained here.
%%
%% All comments are the opinions of their respective authors and are not
%% necessarily endorsed by the IEEE.
%%
%% This work is distributed under the LaTeX Project Public License (LPPL)
%% ( http://www.latex-project.org/ ) version 1.3, and may be freely used,
%% distributed and modified. A copy of the LPPL, version 1.3, is included
%% in the base LaTeX documentation of all distributions of LaTeX released
%% 2003/12/01 or later.
%% Retain all contribution notices and credits.
%% ** Modified files should be clearly indicated as such, including  **
%% ** renaming them and changing author support contact information. **
%%*************************************************************************


% *** Authors should verify (and, if needed, correct) their LaTeX system  ***
% *** with the testflow diagnostic prior to trusting their LaTeX platform ***
% *** with production work. The IEEE's font choices and paper sizes can   ***
% *** trigger bugs that do not appear when using other class files.       ***                          ***
% The testflow support page is at:
% http://www.michaelshell.org/tex/testflow/


\documentclass[conference]{IEEEtran}
\usepackage[utf8]{inputenc}
\usepackage{calrsfs}
% Some Computer Society conferences also require the compsoc mode option,
% but others use the standard conference format.
%
% If IEEEtran.cls has not been installed into the LaTeX system files,
% manually specify the path to it like:
% \documentclass[conference]{../sty/IEEEtran}





% Some very useful LaTeX packages include:
% (uncomment the ones you want to load)


% *** MISC UTILITY PACKAGES ***
%
%\usepackage{ifpdf}
% Heiko Oberdiek's ifpdf.sty is very useful if you need conditional
% compilation based on whether the output is pdf or dvi.
% usage:
% \ifpdf
%   % pdf code
% \else
%   % dvi code
% \fi
% The latest version of ifpdf.sty can be obtained from:
% http://www.ctan.org/pkg/ifpdf
% Also, note that IEEEtran.cls V1.7 and later provides a builtin
% \ifCLASSINFOpdf conditional that works the same way.
% When switching from latex to pdflatex and vice-versa, the compiler may
% have to be run twice to clear warning/error messages.






% *** CITATION PACKAGES ***
%
%\usepackage{cite}
% cite.sty was written by Donald Arseneau
% V1.6 and later of IEEEtran pre-defines the format of the cite.sty package
% \cite{} output to follow that of the IEEE. Loading the cite package will
% result in citation numbers being automatically sorted and properly
% "compressed/ranged". e.g., [1], [9], [2], [7], [5], [6] without using
% cite.sty will become [1], [2], [5]--[7], [9] using cite.sty. cite.sty's
% \cite will automatically add leading space, if needed. Use cite.sty's
% noadjust option (cite.sty V3.8 and later) if you want to turn this off
% such as if a citation ever needs to be enclosed in parenthesis.
% cite.sty is already installed on most LaTeX systems. Be sure and use
% version 5.0 (2009-03-20) and later if using hyperref.sty.
% The latest version can be obtained at:
% http://www.ctan.org/pkg/cite
% The documentation is contained in the cite.sty file itself.






% *** GRAPHICS RELATED PACKAGES ***
%
\ifCLASSINFOpdf
	% \usepackage[pdftex]{graphicx}
	% declare the path(s) where your graphic files are
	% \graphicspath{{../pdf/}{../jpeg/}}
	% and their extensions so you won't have to specify these with
	% every instance of \includegraphics
	% \DeclareGraphicsExtensions{.pdf,.jpeg,.png}
\else
	% or other class option (dvipsone, dvipdf, if not using dvips). graphicx
	% will default to the driver specified in the system graphics.cfg if no
	% driver is specified.
	% \usepackage[dvips]{graphicx}
	% declare the path(s) where your graphic files are
	% \graphicspath{{../eps/}}
	% and their extensions so you won't have to specify these with
	% every instance of \includegraphics
	% \DeclareGraphicsExtensions{.eps}
\fi
% graphicx was written by David Carlisle and Sebastian Rahtz. It is
% required if you want graphics, photos, etc. graphicx.sty is already
% installed on most LaTeX systems. The latest version and documentation
% can be obtained at: 
% http://www.ctan.org/pkg/graphicx
% Another good source of documentation is "Using Imported Graphics in
% LaTeX2e" by Keith Reckdahl which can be found at:
% http://www.ctan.org/pkg/epslatex
%
% latex, and pdflatex in dvi mode, support graphics in encapsulated
% postscript (.eps) format. pdflatex in pdf mode supports graphics
% in .pdf, .jpeg, .png and .mps (metapost) formats. Users should ensure
% that all non-photo figures use a vector format (.eps, .pdf, .mps) and
% not a bitmapped formats (.jpeg, .png). The IEEE frowns on bitmapped formats
% which can result in "jaggedy"/blurry rendering of lines and letters as
% well as large increases in file sizes.
%
% You can find documentation about the pdfTeX application at:
% http://www.tug.org/applications/pdftex





% *** MATH PACKAGES ***
%
%\usepackage{amsmath}
% A popular package from the American Mathematical Society that provides
% many useful and powerful commands for dealing with mathematics.
%
% Note that the amsmath package sets \interdisplaylinepenalty to 10000
% thus preventing page breaks from occurring within multiline equations. Use:
%\interdisplaylinepenalty=2500
% after loading amsmath to restore such page breaks as IEEEtran.cls normally
% does. amsmath.sty is already installed on most LaTeX systems. The latest
% version and documentation can be obtained at:
% http://www.ctan.org/pkg/amsmath





% *** SPECIALIZED LIST PACKAGES ***
%
%\usepackage{algorithmic}
% algorithmic.sty was written by Peter Williams and Rogerio Brito.
% This package provides an algorithmic environment fo describing algorithms.
% You can use the algorithmic environment in-text or within a figure
% environment to provide for a floating algorithm. Do NOT use the algorithm
% floating environment provided by algorithm.sty (by the same authors) or
% algorithm2e.sty (by Christophe Fiorio) as the IEEE does not use dedicated
% algorithm float types and packages that provide these will not provide
% correct IEEE style captions. The latest version and documentation of
% algorithmic.sty can be obtained at:
% http://www.ctan.org/pkg/algorithms
% Also of interest may be the (relatively newer and more customizable)
% algorithmicx.sty package by Szasz Janos:
% http://www.ctan.org/pkg/algorithmicx




% *** ALIGNMENT PACKAGES ***
%
%\usepackage{array}
% Frank Mittelbach's and David Carlisle's array.sty patches and improves
% the standard LaTeX2e array and tabular environments to provide better
% appearance and additional user controls. As the default LaTeX2e table
% generation code is lacking to the point of almost being broken with
% respect to the quality of the end results, all users are strongly
% advised to use an enhanced (at the very least that provided by array.sty)
% set of table tools. array.sty is already installed on most systems. The
% latest version and documentation can be obtained at:
% http://www.ctan.org/pkg/array


% IEEEtran contains the IEEEeqnarray family of commands that can be used to
% generate multiline equations as well as matrices, tables, etc., of high
% quality.




% *** SUBFIGURE PACKAGES ***
%\ifCLASSOPTIONcompsoc
%  \usepackage[caption=false,font=normalsize,labelfont=sf,textfont=sf]{subfig}
%\else
%  \usepackage[caption=false,font=footnotesize]{subfig}
%\fi
% subfig.sty, written by Steven Douglas Cochran, is the modern replacement
% for subfigure.sty, the latter of which is no longer maintained and is
% incompatible with some LaTeX packages including fixltx2e. However,
% subfig.sty requires and automatically loads Axel Sommerfeldt's caption.sty
% which will override IEEEtran.cls' handling of captions and this will result
% in non-IEEE style figure/table captions. To prevent this problem, be sure
% and invoke subfig.sty's "caption=false" package option (available since
% subfig.sty version 1.3, 2005/06/28) as this is will preserve IEEEtran.cls
% handling of captions.
% Note that the Computer Society format requires a larger sans serif font
% than the serif footnote size font used in traditional IEEE formatting
% and thus the need to invoke different subfig.sty package options depending
% on whether compsoc mode has been enabled.
%
% The latest version and documentation of subfig.sty can be obtained at:
% http://www.ctan.org/pkg/subfig




% *** FLOAT PACKAGES ***
%
%\usepackage{fixltx2e}
% fixltx2e, the successor to the earlier fix2col.sty, was written by
% Frank Mittelbach and David Carlisle. This package corrects a few problems
% in the LaTeX2e kernel, the most notable of which is that in current
% LaTeX2e releases, the ordering of single and double column floats is not
% guaranteed to be preserved. Thus, an unpatched LaTeX2e can allow a
% single column figure to be placed prior to an earlier double column
% figure.
% Be aware that LaTeX2e kernels dated 2015 and later have fixltx2e.sty's
% corrections already built into the system in which case a warning will
% be issued if an attempt is made to load fixltx2e.sty as it is no longer
% needed.
% The latest version and documentation can be found at:
% http://www.ctan.org/pkg/fixltx2e


%\usepackage{stfloats}
% stfloats.sty was written by Sigitas Tolusis. This package gives LaTeX2e
% the ability to do double column floats at the bottom of the page as well
% as the top. (e.g., "\begin{figure*}[!b]" is not normally possible in
% LaTeX2e). It also provides a command:
%\fnbelowfloat
% to enable the placement of footnotes below bottom floats (the standard
% LaTeX2e kernel puts them above bottom floats). This is an invasive package
% which rewrites many portions of the LaTeX2e float routines. It may not work
% with other packages that modify the LaTeX2e float routines. The latest
% version and documentation can be obtained at:
% http://www.ctan.org/pkg/stfloats
% Do not use the stfloats baselinefloat ability as the IEEE does not allow
% \baselineskip to stretch. Authors submitting work to the IEEE should note
% that the IEEE rarely uses double column equations and that authors should try
% to avoid such use. Do not be tempted to use the cuted.sty or midfloat.sty
% packages (also by Sigitas Tolusis) as the IEEE does not format its papers in
% such ways.
% Do not attempt to use stfloats with fixltx2e as they are incompatible.
% Instead, use Morten Hogholm'a dblfloatfix which combines the features
% of both fixltx2e and stfloats:
%
% \usepackage{dblfloatfix}
% The latest version can be found at:
% http://www.ctan.org/pkg/dblfloatfix




% *** PDF, URL AND HYPERLINK PACKAGES ***
%
%\usepackage{url}
% url.sty was written by Donald Arseneau. It provides better support for
% handling and breaking URLs. url.sty is already installed on most LaTeX
% systems. The latest version and documentation can be obtained at:
% http://www.ctan.org/pkg/url
% Basically, \url{my_url_here}.




% *** Do not adjust lengths that control margins, column widths, etc. ***
% *** Do not use packages that alter fonts (such as pslatex).         ***
% There should be no need to do such things with IEEEtran.cls V1.6 and later.
% (Unless specifically asked to do so by the journal or conference you plan
% to submit to, of course. )


% correct bad hyphenation here
\hyphenation{op-tical net-works semi-conduc-tor}


\begin{document}
%
% paper title
% Titles are generally capitalized except for words such as a, an, and, as,
% at, but, by, for, in, nor, of, on, or, the, to and up, which are usually
% not capitalized unless they are the first or last word of the title.
% Linebreaks \\ can be used within to get better formatting as desired.
% Do not put math or special symbols in the title.
\title{ERA Streaming\\ Assistenza da remoto tramite Streaming}


% author names and affiliations
% use a multiple column layout for up to three different
% affiliations
\author{\IEEEauthorblockN{Paolo Bonato}
\IEEEauthorblockA{Università degli Studi di Padova \\
Email: paolo.bonato.12@gmail.com}
\and
\IEEEauthorblockN{Tommaso Padovan}
\IEEEauthorblockA{Università degli Studi di Padova \\
Email: tommaso.pado@gmail.com}}

% conference papers do not typically use \thanks and this command
% is locked out in conference mode. If really needed, such as for
% the acknowledgment of grants, issue a \IEEEoverridecommandlockouts
% after \documentclass

% for over three affiliations, or if they all won't fit within the width
% of the page, use this alternative format:
% 
%\author{\IEEEauthorblockN{Michael Shell\IEEEauthorrefmark{1},
%Homer Simpson\IEEEauthorrefmark{2},
%James Kirk\IEEEauthorrefmark{3}, 
%Montgomery Scott\IEEEauthorrefmark{3} and
%Eldon Tyrell\IEEEauthorrefmark{4}}
%\IEEEauthorblockA{\IEEEauthorrefmark{1}School of Electrical and Computer Engineering\\
%Georgia Institute of Technology,
%Atlanta, Georgia 30332--0250\\ Email: see http://www.michaelshell.org/contact.html}
%\IEEEauthorblockA{\IEEEauthorrefmark{2}Twentieth Century Fox, Springfield, USA\\
%Email: homer@thesimpsons.com}
%\IEEEauthorblockA{\IEEEauthorrefmark{3}Starfleet Academy, San Francisco, California 96678-2391\\
%Telephone: (800) 555--1212, Fax: (888) 555--1212}
%\IEEEauthorblockA{\IEEEauthorrefmark{4}Tyrell Inc., 123 Replicant Street, Los Angeles, California 90210--4321}}




% use for special paper notices
%\IEEEspecialpapernotice{(Invited Paper)}




% make the title area
\maketitle

% As a general rule, do not put math, special symbols or citations
% in the abstract
\begin{abstract}
Questo \textit{report} descrive l'applicazione ERAStreaming e le sue fasi di sviluppo. Si tratta di una applicazione 
sviluppata nativamente per \textit{smartphone} Android che permettere ottenere assistenza da 
remoto tramite \textit{Streaming}. Fornisce inoltre un \textit{hub} sociale che permette agli utenti
di entrare in contatto per offrire o riceve assistenza.\\Per fornire alcuni servizi in tempo reale
come la \textit{chat} o le recensioni degli utenti \textit{ERAStreaming} usa un database noSql real-time.
Per quanto riguarda lo \textit{Streaming} invece fa uso di un server dedicato con un protocollo di tipo
RTMP creato appositamente dagli autori.
\end{abstract}

% no keywords




% For peer review papers, you can put extra information on the cover
% page as needed:
% \ifCLASSOPTIONpeerreview
% \begin{center} \bfseries EDICS Category: 3-BBND \end{center}
% \fi
%
% For peerreview papers, this IEEEtran command inserts a page break and
% creates the second title. It will be ignored for other modes.
\IEEEpeerreviewmaketitle



\section{Introduzione}
	Lo scopo del progetto è quello di permettere all'utenza di riceve assistenza
	remota per un ventaglio quanto più possibile vasto di ambiti.
	Chi ha bisogno di supporto per una qualsiasi mansione avrà la possibilità
	di collegarsi con un tecnico competente che potrà, tramite \textit{streaming} video,
	supervisionare il lavoro e dare istruzioni per portarlo a termine.

	\subsection{Target}
		Il target naturale per questo genere di applicazioni dovrebbe essere quello degli
		\textit{smart-glasses}, o comunque dispositivi \textit{wearable}. Essi infatti
		permetto di operare a mani libere e di avere un punto di vista preferenziale per
		il tutor.\\
		Durante le fasi preliminari di questo progetto, pertanto, è stato condotto uno studio
		di fattibilità, riguardante in particolare i dispositivi \textit{wearable}: essi si sono 
		però dimostrati una tecnologia ancora non sufficientemente matura. dal punto di vista
		\textit{Hardware}:
		\begin{itemize}
			\item Durata insufficiente della batteria.
			\item Surriscaldamento.
			\item Limitata potenza di calcolo, e quindi di compressione dell'immagine.
			\item Scarsa qualità della fotocamera.
		\end{itemize}
		Ma anche \textit{Software}:
		\begin{itemize}
			\item La maggior parte dei dispositivi supporta una versione ridotta di Android KitKat 4.4,
			\item La maggior parte dei dispositivi non supporta le nuove API per gestire la camera, e 
			fa utilizzo di API precedenti deprecate negli altri dispositivi.
		\end{itemize}
		Per questi motivi dunque si è scelto di sviluppare l'applicazione per smartphone, lasciando
		il \textit{porting} o lo sviluppo di una \textit{companion-app} per un momento futuro.

	\subsection{Motivazioni e Contesto}
		Il \textit{concept} applicazione è nato per utilizzo in ambito professionale,
		ad esempio per diminuire il numero di trasferte per riparazione di macchinari oppure per 
		riceve veloce supporto in operazioni critiche in cui non c'è tempo o possibilità di attendere
		l'arrivo di un tecnico specializzato. In generale nel contesto dell'industria 4.0 servizi di questo
		genere saranno sempre più richiesti in quanto sempre più dispositivi saranno connessi alla rete
		e quindi manutenuti da remoto, per gli altri di certo, ci sarà necessità di una assistenza altrettanto
		immediata.\\
		Tale servizio, però, potrebbe prendere piede anche in ambito \textit{consumer}: sostituire lo schermo dello smartphone, 
		sostituire la batteria danneggiata di un laptop, riparare un piccolo elettrodomestico e molto altro
		sono esigenze all'ordine del giorno.
		Per tali motivazioni è stato deciso di intraprendere questo progetto sviluppando non solo un sistema
		efficiente di \textit{streaming}, ma anche una interfaccia semplice ed immediata per consentire a tutte
		le fasce di utenza di accedervi. Inoltre sarà possibile in futuro salvare le sessioni di assistenza sul 
		server creando così un database di contenuti e di \textit{know-how} autogenerati.
				

% no \IEEEPARstart
% You must have at least 2 lines in the paragraph with the drop letter
% (should never be an issue)




 
\section{Proposta di soluzione}
	La parte di \textit{hub} e quella di streaming sono state sviluppate in maniera completamente indipendente.
	Si è deciso di mantenerle distaccate il più possibile per due ragioni principali:
	\begin{itemize}
		\item La parte di streaming deve essere un prodotto indipendente: alcuni \textit{wearable} potrebbero
		non avere uno schermo (e.g. i \textit{Google Glass}) o potrebbero avere altre limitazioni.
		\item Le due componenti hanno esigenze computazionali molto diverse: la prima ha bisogno di indicizzare e gestire velocemente
		grandi liste di danti, mentre la seconda deve avere una infrastruttura per gestire lunghi \textit{burst} e di uno spazio
		di archiviazione maggiore.
	\end{itemize}
	Nelle sezioni seguenti verranno esposte nel dettaglio queste due componenti.

	\subsection{Hub}
		Questa macro-componente del sistema si occupa di interfacciarsi con l'utente, raccogliere informazioni, far incontrare l'utente con
		il tutor adatto e passare i dati dei dispositivi da far comunicare alla macro-componente di streaming.
		Inoltre fornisce una \textit{real-time chat} ed un sistema di \textit{rating}.\\
		Un aspetto centrale di questa componente è il coinvolgimento dell'utenza. Per questo si è cercato di tenere il \textit{client} il quanto
		più possibile semplice e reattivo.\\
		La tecnologia predominante usata in questa parte è la \textit{suite Firebase}.
		
		\subsubsection{Firebase}
		\begin{itemize}
			\item \textbf{Realtime - noSQL}
				\textit{Firebase Database} è il database consigliato dalle \textit{best-practices} Android. È un database di tipo noSQL
				che salva i dati in formato \textit{json}. Ha un ottima integrazione con il \textit{framework} Android in quando permette
				di salvare e leggere dati direttamente come oggetti Java.\\
				Non solo permette, ma costringe	lo sviluppatore a leggere i dati in maniera asincrona tramite \textit{listener design pattern}. Ciò si riflette
				in una ottima reattività del prodotto, in quanto tutti i dati sono aggiornati in maniera istantanea non appena essi
				vengono modificati sul database senza bisogno di \textit{refresh} o altre azioni da parte dell'utente.\\
			
			\item \textbf{Serverless}
				Questa \textit{suite} incoraggia una architettura \textit{serverless}: ovvero non necessita di un server fisico e statico.
				Il database e qualsiasi altra infrastruttura sono \textit{hostati} dai server Google e la quantità di spazio disco, 
				banda, CPU etc. viene modificata dinamicamente a seconda del carico a cui l'applicazione è sottoposta.
		\end{itemize}
		\subsubsection{Struttura generale}
			L'applicazione è stata realizzata nativamente per Android. L'aspetto centrale è la figura dell'utente che ha il suo profilo
			con nome, cognome, email, foto, biografia e lista di competenze. La caratterizzazione dell'utente potrà essere facilmente
			incrementata nel futuro perché è stata implementata con la massima flessibilità.\\
			Ogni utente inoltre può cercare nel database altri profili in base alle competenze necessarie, ma anche per nome o email.\\
			Una volta trovato il tutor cercato è possibile aggiungerlo alla propria rete di contatti: una volta che la richiesta viene
			accettata sarà possibile accordarsi tramite chat e poi far partire una sessione di supporto video. Ovviamente è possibile
			anche rimuovere un contatto indesiderato.\\
			Ogni utente può modificare il proprio profilo in ogni momento; per migliorare la qualità dei profili disponibili
			al primo login viene chiesto di fornire alcuni dati personali.\\
			Per mantenere una buona qualità delle sessioni di assistenza, è messa a disposizione una funzione di recensione degli utenti.

		\subsubsection{Realtime chat}
			La chat in tempo reale è uno dei più importanti requisiti di questa parte, deve essere semplice, immediata e fruibile.
			La soluzione proposta usa nativamente \textit{Firebase} sfruttando la sua natura real-time.\\
			Ad ogni coppia di utenti viene assegnato un codice univoco di una stanza di chat che non è altro che un nodo particolare
			dell'albero \textit{json} presente nel database a cui entrambi hanno permessi di lettura e scrittura. Per inviare un nuovo
			messaggio è sufficiente quindi che il client faccia la \textit{push} di un nuovo messaggio sul database. Per 
			visualizzare la lista dei messaggi invece basta seguire ancora una volta il \textit{listener design pattern} e registrare
			entrambi gli utenti alla lista degli ascoltatori di una particolare stanza di chat, il sistema Android quindi provvederà
			automaticamente ed in tempo reale a tenere aggiornata la pagina con l'elenco dei messaggi.

		\subsubsection{Rete dei contatti}
			La relazione di "amicizia" o di "contatto" tra coppie di utenti è un classico \textit{use-case} per i database di tipo
			relazionale; in un database di tipo noSQL apre una problematica molto diversa. \textit{Firebase} non fornisce
			un servizio di database relazionale, quindi è necessario usare una struttura diversa e, inevitabilmente, introdurre
			ridondanza.\\
			Si prenda in considerazione il problema di trovare tutti gli ID degli "amici" di un determinato utente:
			se il database fosse strutturato con un nodo \textit{json} che contiene coppie di utenti tra loro "amici" (come in un classico SQL)
			per ottenere tale lista sarebbe necessario un tempo lineare nella dimensione del database, mentre aggiungendo ad ogni utente
			la lista degli \texttt{id}	 dei propri amici è possibile ottenere questa lista in \texttt{O(1)}.\\
			Per gestire in maniera controllata questa ridondanza ed evitare possibili anomalie è necessario del codice
			lato server. \textit{Firebase} fornisce una fuzionalità chiamata \textit{Firebase Functions} che permette esattamente
			di fare questo; sempre in un contesto serverless è possibile caricare del codice \textit{nodeJS} che viene eseguito
			in riposta ad eventi legati al database oppure in riposta a delle richieste \textit{GET} ad un determinato 
			\textit{end-point http}.

		\subsubsection{Notifiche}
			Le notifiche sono un altro aspetto molto importante per questo genere di applicazioni. Ad esempio devono essere mandate
			all'utente quando riceve un nuovo messaggio di chat, oppure quando ha un nuova richiesta di amicizia e così via.\\
			Ancora una volta per la gestione delle notifiche push è stato scelto un servizio Google denominato \textit{Firebase Cloud Messagging} o FCM.
			Ad ogni utente \textit{Firebase} viene associato automaticamente un token identificativo, esso può essere usato per inviare
			\textit{push message}. Inoltre gruppi di utenti possono sottoscriversi a \textit{topic} e possono ricevere notifiche push
			riguardo ai loro argomenti di interesse.\\
			Nel prodotto presentato questa funzione è stata implementata usando congiuntamente \textit{Firebase Database}, \textit{Firebase Functions} e il \textit{Cloud Messagging}.
			Non è indicato che un utente abbia accesso al token FCM di un altro, perché creerebbe grossi problemi di sicurezza; pertanto
			l'invio di notifiche push da parte di un utente $\mathcal{A}$ verso un altro utente $\mathcal{B}$ (come nel caso di una richiesta di amicizia)
			è stato strutturato come segue:
			\begin{itemize}
				\item \textbf{Push request:} L'utente $\mathcal{A}$ aggiunge un nodo (contenente tutti i dati della notifica) al database in una tabella speciale ad esempio
					\texttt{/pushQueues/friendRequests}. È una semplice scrittura su database; $\mathcal{A}$ a questo punto non deve fare altro.
				\item \textbf{Sanity check:} Il nodo sopra citato è "speciale" in quando ha una \textit{Firebase Function} registrata come \textit{observer}.
					Questa funzione ad ogni nuova scrittura in \texttt{pushQueues} legge i dati della richiesta di notifica push, verifica che $\mathcal{A}$ abbia
					i corretti permessi per mandare una notifica a $\mathcal{B}$ e in caso affermativo legge nel database l'FCM-token di $\mathcal{B}$.
				\item \textbf{Push notification:} Se la funzione del passo precedente termina con esito positivo allora viene chiamato il servizio FCM, che si occupa
					di inviare un \textit{Firebase Message} al corretto dispositivo (o gruppo di dispositivi).
				\item \textbf{Ricezione:} Il client installato sul dispositivo dell'utente $\mathcal{B}$ riceve il messaggio tramite un \textit{service} e notifica
					l'utente in base alle sue impostazioni sulle notifiche.
			\end{itemize}



	\subsection{Streaming}
		Per lo streaming real-time di video e audio non sono state reperite librerie open-source gratuite.
		Inoltre l'applicazione si propone di potersi interfacciare in futuro con devices quali smart-glasses,
		quindi è richiesto un profondo controllo su tale funzionalità. Il tutto è stato realizzato utilizzando API native di Android.

		\subsubsection{Risorse utilizzate}
			Per la codifica e decodifica di audio e video Android fornisce la classe \texttt{MediaCodec}, capace di interfacciarsi con coder e decoder sia software che hardware e permette di specificare il formato di coding desiderato e di impostarne i parametri per regolare il livello di compressione. Per la comunicazione real-time si sono scelti gli standard h.264 AVC per il video e AAC per l'audio, in quanto permettono di comprimere molto i dati per diminuire il carico di rete.
			Per utilizzare la fotocamera si è deciso di esplorare e sperimentare le nuove API Camera2 di Android, mentre l'audio viene gestito tramite le classi AudioRecord e AudioTrack rispettivamente per cattura e riproduzione.
			La compressione richiede un carico di lavoro molto elevato e per tempo prolungato in quanto bisogna costantemente fornire dei buffer ai coder e utilizzare i buffer ritornati, il tutto in tempo reale. A questo scopo i costrutti Android per gestire operazioni asincrone non sono risultati sufficienti e si è deciso di utilizzare i più "leggeri" Thread di Java. 
			Per il networking si è deciso di operare tramite Socket Java utilizzando come protocollo per lo streaming un RTP-like da noi implementato. Per la connessione tramite server è già disposto un cambiamento, quindi tutta la parte di networking è realizzata seguendo il \textit{design pattern Strategy}.

		\subsubsection{Funzionalità}
			Lo streaming realtime è funzionante e consente una comunicazione fluida su rete locale, non è stato ancora testato utilizzando un server remoto. L'interfaccia utente mette a disposizione il video ricevuto e una piccola finestra per visualizzare ciò che sta riprendendo. Per il momento si è scelto di mettere a disposizione solo la fotocamera frontale in quanto lo scopo è quello di mostrare ciò che l'utente sta facendo in fronte a lui, in futuro sarà implementato anche il cambio fotocamera (quando disponibile).
			La comunicazione è gestita tramite protocollo TCP il quale, nonostante sia più lento rispetto ad UDP, semplifica enormemente la connettività ed è più indicato per dispositivi mobili. Anche se non è stato ancora implementato lato utente è già possibile modificare la qualità del video per rispondere alle costrizioni della rete.

		\subsubsection{Problemi Riscontrati}
			\begin{itemize}
				\item Implementare una videochiamata real-time utilizzando API di basso livello si è dimostrata un operazione difficile ma al termine lascia un controllo completo e rende possibile la portabilità su tecnologie wearable.
				\item Android risulta avere un problema irrisolto per quanto riguarda la latenza dell'audio, portando ad una round-trip latency (tempo intercorso tra l'acquisizione e la riproduzione del suono) molto più elevata rispetto ad altri sistemi.
			\end{itemize}





\section{Conclusioni}
	

	\subsection{Sviluppi futuri}
		Questo prodotto apre la strada per molti sviluppi futuri.\\
		\subsubsection{Wearable}
			Primo su tutti c'è il \textit{porting} per un adeguato dispositivo \textit{wearable}; sarà necessario
			tenere conto della limitatezza dell'Hardware a disposizione e del diverso modo di interfacciarsi con l'utente.
			Un possibile sviluppo in questo senso potrebbe essere una \textit{companion app}, ovvero un'applicazione che
			deve coesistere con l'app per smartphone per integrare alcune funzionalità. Ad esempio si potrebbe sfruttare
			la fotocamera di uno \textit{smarglass} per registrare il video lasciando le mani libere all'operatore.\\

		\subsubsection{Tools per l'assistente}
			Una migliore gestione da parte dell'assistente della sessione potrebbe essere un valore aggiunto notevole.
			Potrebbero essere inseriti alcuni strumenti per il tutor come:
			\begin{itemize}
				\item Possibilità di far comparire a video appunti scritti per l'assistito.
				\item Tracciare grafici o disegni/schemi a mano libera.
				\item Poter scorrere avanti ed indietro	il video per trovare alcuni fermo immagine.
				\item Tracciare/evidenziare parti del video live (ad esempio per far capire quale vite allentare)
					e rappresentare questi dati a video per l'assistito, o addirittura in realtà aumentata in caso
					di \textit{wearable}.
			\end{itemize}

		\subsubsection{Migliorie hub}
			Alcuni aspetti della parte sociale dell'applicazione devono essere migliorati. Ad esempio serve
			una maglia più fine per le categorie di competenza e un modo più user-friendly di raccoglierle.
			Possono essere aggiunti aspetti di apprendimento automatico per suggerire agli utenti altri
			utenti con cui probabilmente vorrebbero entrare in contatto.

	\subsection{Risultati}
		I risultati sono soddisfacenti. La parte di \textit{hub} si è rivelata semplice ma immediata, i dati sugli 
		utenti vengono aggiornati in maniera real-time e la chat è perfettamente funzionante con un ritardo quasi
		inesistente.\\
		La parte di streaming ha un ottimo funzionamento e riesce a trasmettere video ad una buona qualità senza
		ritardi apprezzabili e senza grande perdita di pacchetti. Le conversazioni audio sono anch'esse di buona
		qualità e in ogni caso è possibile avere una sessione di assistenza completa e chiara.
	




% An example of a floating figure using the graphicx package.
% Note that \label must occur AFTER (or within) \caption.
% For figures, \caption should occur after the \includegraphics.
% Note that IEEEtran v1.7 and later has special internal code that
% is designed to preserve the operation of \label within \caption
% even when the captionsoff option is in effect. However, because
% of issues like this, it may be the safest practice to put all your
% \label just after \caption rather than within \caption{}.
%
% Reminder: the "draftcls" or "draftclsnofoot", not "draft", class
% option should be used if it is desired that the figures are to be
% displayed while in draft mode.
%
%\begin{figure}[!t]
%\centering
%\includegraphics[width=2.5in]{myfigure}
% where an .eps filename suffix will be assumed under latex, 
% and a .pdf suffix will be assumed for pdflatex; or what has been declared
% via \DeclareGraphicsExtensions.
%\caption{Simulation results for the network.}
%\label{fig_sim}
%\end{figure}

% Note that the IEEE typically puts floats only at the top, even when this
% results in a large percentage of a column being occupied by floats.


% An example of a double column floating figure using two subfigures.
% (The subfig.sty package must be loaded for this to work.)
% The subfigure \label commands are set within each subfloat command,
% and the \label for the overall figure must come after \caption.
% \hfil is used as a separator to get equal spacing.
% Watch out that the combined width of all the subfigures on a 
% line do not exceed the text width or a line break will occur.
%
%\begin{figure*}[!t]
%\centering
%\subfloat[Case I]{\includegraphics[width=2.5in]{box}%
%\label{fig_first_case}}
%\hfil
%\subfloat[Case II]{\includegraphics[width=2.5in]{box}%
%\label{fig_second_case}}
%\caption{Simulation results for the network.}
%\label{fig_sim}
%\end{figure*}
%
% Note that often IEEE papers with subfigures do not employ subfigure
% captions (using the optional argument to \subfloat[]), but instead will
% reference/describe all of them (a), (b), etc., within the main caption.
% Be aware that for subfig.sty to generate the (a), (b), etc., subfigure
% labels, the optional argument to \subfloat must be present. If a
% subcaption is not desired, just leave its contents blank,
% e.g., \subfloat[].


% An example of a floating table. Note that, for IEEE style tables, the
% \caption command should come BEFORE the table and, given that table
% captions serve much like titles, are usually capitalized except for words
% such as a, an, and, as, at, but, by, for, in, nor, of, on, or, the, to
% and up, which are usually not capitalized unless they are the first or
% last word of the caption. Table text will default to \footnotesize as
% the IEEE normally uses this smaller font for tables.
% The \label must come after \caption as always.
%
%\begin{table}[!t]
%% increase table row spacing, adjust to taste
%\renewcommand{\arraystretch}{1.3}
% if using array.sty, it might be a good idea to tweak the value of
% \extrarowheight as needed to properly center the text within the cells
%\caption{An Example of a Table}
%\label{table_example}
%\centering
%% Some packages, such as MDW tools, offer better commands for making tables
%% than the plain LaTeX2e tabular which is used here.
%\begin{tabular}{|c||c|}
%\hline
%One & Two\\
%\hline
%Three & Four\\
%\hline
%\end{tabular}
%\end{table}


% Note that the IEEE does not put floats in the very first column
% - or typically anywhere on the first page for that matter. Also,
% in-text middle ("here") positioning is typically not used, but it
% is allowed and encouraged for Computer Society conferences (but
% not Computer Society journals). Most IEEE journals/conferences use
% top floats exclusively. 
% Note that, LaTeX2e, unlike IEEE journals/conferences, places
% footnotes above bottom floats. This can be corrected via the
% \fnbelowfloat command of the stfloats package.








% conference papers do not normally have an appendix


% use section* for acknowledgment
%\section*{Acknowledgment}


%The authors would like to thank...





% trigger a \newpage just before the given reference
% number - used to balance the columns on the last page
% adjust value as needed - may need to be readjusted if
% the document is modified later
%\IEEEtriggeratref{8}
% The "triggered" command can be changed if desired:
%\IEEEtriggercmd{\enlargethispage{-5in}}

% references section

% can use a bibliography generated by BibTeX as a .bbl file
% BibTeX documentation can be easily obtained at:
% http://mirror.ctan.org/biblio/bibtex/contrib/doc/
% The IEEEtran BibTeX style support page is at:
% http://www.michaelshell.org/tex/ieeetran/bibtex/
%\bibliographystyle{IEEEtran}
% argument is your BibTeX string definitions and bibliography database(s)
%\bibliography{IEEEabrv,../bib/paper}
%
% <OR> manually copy in the resultant .bbl file
% set second argument of \begin to the number of references
% (used to reserve space for the reference number labels box)


%\begin{thebibliography}{1}
%
%\bibitem{IEEEhowto:kopka}
%H.~Kopka and P.~W. Daly, \emph{A Guide to \LaTeX}, 3rd~ed.\hskip 1em plus
%	0.5em minus 0.4em\relax Harlow, England: Addison-Wesley, 1999.
%
%\end{thebibliography}
%
%
%
%
%% that's all folks
\end{document}


